\section{Einleitung}
\label{sec:einleitung}

Stacking-Probleme beschreiben Situationen, in welchen es darum geht, eine Menge von Items, häufig Container,
unter Berücksichtigung bestimmter Nebenbedingungen in einer Lagerfläche zu positionieren. Das \textquote{Stacking}
bezieht sich auf die Tatsache, dass das Stapeln der Items essenziell ist, um möglichst viele Items
auf geringer Fläche unterzubringen. In der Praxis treten solche Probleme zum Beispiel im Umfeld von Lagerhallen
und Container-Terminals auf.\newline
Das System des intermodalen Frachttransports durch Container spielt eine zentrale Rolle im internationalen Handel.
Der schnellste und kosteneffektivste Weg, allgemeine Frachten zu verschicken, ist typischerweise der
Containertransport (vgl. \citet{Briskorn2018}).
Mit steigendem Durchsatz gewinnt die effiziente Verwaltung der Container an Bedeutung.
Der Containerumschlag am Hamburger Hafen hat sich beispielsweise zwischen 1990 und 2017
annähernd verdreifacht (vgl. \citet{Port_of_Hamburg}).
Folglich stellt das sinnvolle Ein- und Auslagern der Container eine immer größere Herausforderung dar und
es herrscht ein großer Bedarf, diese Prozesse zu optimieren.

Bisher existieren kaum Veröffentlichungen, welche effiziente Lösungsverfahren für Stack\-/ing-Probleme thematisieren.
Insbesondere die effiziente Minimierung der Transportkosten, die beim Verladen der Items von den Fahrzeugen,
mit welchen sie geliefert werden, in die Lagerfläche, entstehen, hat in der Literatur noch keine Beachtung gefunden,
obwohl diese auch aus praktischer Perspektive von Relevanz ist. Ziel dieser Arbeit ist es dementsprechend,
ebensolche heuristische Verfahren für verschiedene Varianten von Stacking-Problemen zu entwickeln, bei denen der Fokus auf
Zulässigkeit und geringer Laufzeit liegt.\newline
Es gibt deutlich mehr Veröffentlichungen zu Unloading-, also Auslagerungs-Szenarien und Szenarien, bei denen simultan Items
ein- und ausgelagert werden, als zu reinen Loading-Problemen, welche Gegenstand dieser Arbeit sind.
Der folgende Literaturüberblick beschränkt sich auf diese wenigen Veröffentlichungen, welche zusätzlich bestimmte Konzepte,
die im Verlauf der Arbeit verwendet oder erweitert werden, etablieren, oder themennahe Problemstellungen behandeln.

Zunächst werden, im Anschluss an den in Kapitel \ref{sec:literature_review} folgenden Literaturüberblick,
in Kapitel \ref{sec:stacking_problems} Stacking-Probleme im Allgemeinen eingeführt und anschließend formal definiert.
Da­r­auf­fol­gend geht es in Kapitel \ref{sec:test_data} um die Testinstanzen, welche generiert werden,
um die entwickelten Heuristiken testen und miteinander vergleichen zu können.
Ebenfalls dem experimentellen Vergleich dienen die in Kapitel \ref{sec:mip_formulations} eingeführten MIP-Formulierungen, welche durch Ermittlung der optimalen Zielfunktionswerte eine Beurteilung der Lösungsqualität der Heuristiken ermöglichen.
Nachdem Kapitel \ref{sec:lower_bounds} untere Schranken für Stacking-Probleme thematisiert,
werden in Kapitel \ref{sec:constructive_heuristics} schließlich die entwickelten konstruktiven Heuristiken vorgestellt und die Ergebnisse der experimentellen Vergleiche präsentiert. Anschließend wird in Kapitel \ref{sec:post_optimization} ein Verbesserungsverfahren beschrieben, welches von den Heuristiken aus Kapitel \ref{sec:constructive_heuristics} generierte Lösungen als Eingabe erhält. Dieses Verfahren hat zum Ziel, die Initiallösungen basierend auf einer
lokalen Suche zu verbessern. Abschließend wird in Kapitel \ref{sec:conclusion} ein Fazit formuliert.

\vfill
\pagebreak

\section{Literaturüberblick}
\label{sec:literature_review}

\citet{Kim2000} betrachten ein Loading-Problem, bei welchem es darum geht, die Positionierung eintreffender
Container basierend auf ihrem Gewicht vorzunehmen. Es wird ein Ansatz mit dynamischer Programmierung beschrieben,
welcher die Position der Container in der Lagerfläche bestimmt und dabei die Anzahl der erwarteten
Relocation-Operationen\footnote{Umsortierungen innerhalb der Stacks.} minimiert.
Basierend auf den damit generierten Optimallösungen der betrachteten Instanzen wird ein Decision-Tree, welcher Echtzeit-Entscheidungen ermöglicht, entwickelt.
Dementsprechend handelt es sich bei diesem Ansatz durchaus um ein effizientes Lösungsverfahren für Stacking-Probleme.
Es wird mit der Minimierung der Anzahl der erwarteten Relocation-Operationen allerdings eine andere Zielfunktion
als in dieser Arbeit betrachtet.\newline
\citet{Kang2006} beschäftigen sich mit einem ähnlichen Problem und beschreiben einen Simulated-Annealing-Ansatz,
welcher Stacking-Strategien für Container mit ungewissen Gewichtsinformationen bereitstellt.

Eine weitere themennahe Veröffentlichung liefern \citet{Delgado2012}, welche sich mit einem
Loading-Problem beschäftigen, bei dem ein Containerschiff mit einer Menge an Items beladen wird.
Basierend auf der Stabilität des Schiffs sind für jeden Stack Gewichts- und Höhenbeschränkungen gegeben.
Da einige Container eine Energiequelle benötigen, sind zusätzlich bestimmte Restriktionen bezüglich der
Positionierung gegeben.
Die betrachtete Zielfunktion entspricht der gewichteten Summe vierer Zielfunktionen.
Es werden \textquote{Overstows}\footnote{Container, welche auf einem anderen Container mit früherem Auslagerungszeitpunkt platziert sind.} minimiert und das Platzieren von Containern mit unterschiedlichen Zielhäfen
im selben Stack, die Eröffnung neuer Stacks sowie das Platzieren eines Containers,
welcher keine Energiequelle benötigt, an einer Position, an welcher eine solche vorliegt, vermieden.
Durch eine Reduktion vom \textsc{BinPacking}-Problem wird gezeigt,
dass die Minimierung der Anzahl verwendeter Stacks NP-schwer ist.
Die Restriktionen bezüglich der Positionierung bestimmter Items sind auch Teil der Nebenbedingungen,
welche in dieser Arbeit betrachtet werden.

Im Zusammenhang mit Bahn-Terminals betrachtet \citet{Jaehn2013} ein Loading-Problem,
bei welchem Items von einer Anzahl von Zügen, welche gleichzeitig eintreffen,
in eine Lagerfläche verladen werden, welche aus parallelen \textquote{Lanes} mit limitierter Kapazität besteht.
Es wird zunächst die NP-Schwere zweier Modelle gezeigt, woraufhin heuristische Algorithmen vorgestellt werden,
welche mit realen Daten getestet werden. Eine der dort entwickelten Heuristiken kommt sogar in der Praxis
in einem Hafen zum Einsatz. Auch wenn es sich prinzipiell um eine der Problemstellung in dieser
Arbeit nicht ferne Ausgangssituation handelt, gibt es doch einige grundlegende Unterschiede.
Unter anderem stellt das Stapeln von Containern im dort betrachteten Szenario eher die Ausnahme dar.

\vfill
\pagebreak

\citet{Bruns2015} veröffentlichen erste Ergebnisse zur Komplexität verschiedener Stacking-Probleme.
Daraus stammt unter anderem die Erkenntnis, dass eine der in dieser Arbeit betrachteten Problemstellungen stark NP-vollständig ist. Des Weiteren werden Polynomialzeit-Algorithmen für einige Stacking-Probleme vorgestellt,
welche zum Teil in den im Verlauf dieser Arbeit entwickelten Heuristiken Anwendung finden.\newline
Gleichermaßen Verwendung in dieser Arbeit finden die von \citet{Le2016} entwickelten
MIP-Formulierungen\footnote{MIP: Mixed-Integer-Programming.} zur Lösung von Loading-Problemen,
bei welchen es darum geht, die Anzahl der verwendeten Stacks zu minimieren. Diese ermöglichen in angepasster Form einen experimentellen Vergleich mit den entwickelten Heuristiken in Bezug auf die Lösungsqualität.

Diese Beispiele stehen exemplarisch dafür, dass es sich bei Storage-Loading- bzw. Stacking-Problemen um
ein aktives Forschungsgebiet handelt, welches auch aus praktischer Perspektive von großer Relevanz ist.
Außerdem verdeutlichen sie, dass der Fokus bisher in der Regel auf der Optimierung anderer Zielfunktionen lag.

\vfill
\pagebreak
