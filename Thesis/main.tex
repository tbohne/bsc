% Vorlage der AG KO fuer Abschlussarbeiten zum doppelseitigen Druck
\documentclass[%
a4paper,
12pt,
parskip=half,
numbers=noenddot,
twoside,
abstract=on,
headings=standardclasses,
headings=big,
]{scrartcl}

\usepackage[english,main=ngerman]{babel}

\usepackage[margin=2.8cm,bottom=2.8cm]{geometry}
\usepackage[utf8]{inputenc}
\usepackage[T1]{fontenc}

\usepackage{lmodern}

\usepackage{graphicx}
\usepackage{subcaption}
% \usepackage{array}
% \usepackage{longtable}
% \usepackage{verbatim}

\usepackage{booktabs}
\usepackage[table]{xcolor}
\usepackage[vlined,ruled]{algorithm2e}
%\usepackage{tabularx}

\usepackage{amsmath}
\usepackage{amssymb}
\usepackage{amsthm}
% \usepackage{comment}
% \usepackage{enumitem}
% \usepackage[binary-units=true]{siunitx}
% \usepackage{thmtools}
\usepackage{csquotes}

\usepackage{float}
\usepackage{tikz}
\usetikzlibrary{automata,positioning}
\usepackage{mathtools}
\usepackage{algpseudocode}

\DeclarePairedDelimiter{\ceil}{\lceil}{\rceil}
\definecolor{mygreen}{RGB}{1,135,1}

\usepackage[]{hyperref}
\hypersetup{
  colorlinks,
  linkcolor={red!50!black},
  citecolor={blue!90!black},
  urlcolor={blue!50!black}
}

% \usepackage{tikz}
% \usetikzlibrary{arrows.meta}
% \usetikzlibrary{shapes}
% \usetikzlibrary{calc}
% \usetikzlibrary{backgrounds}
% \usetikzlibrary{positioning}
% \usetikzlibrary{decorations.pathmorphing}


%---------------------------------------------------------------------%
% Floats

\def\textfraction{.01}
\def\topfraction{.8}
\def\bottomfraction{.8}
\def\floatpagefraction{.8}


%---------------------------------------------------------------------%
% Bibiliography

\usepackage[
style=numeric,
%backend=biber,      % auf Linux
backend=bibtex,    % auf Windows und Linux
natbib
]{biblatex}

\DeclareFieldFormat[
article,inbook,incollection,inproceedings,patent,thesis,unpublished
]{citetitle}{#1}
\DeclareFieldFormat[
article,inbook,incollection,inproceedings,patent,thesis,unpublished
]{title}{#1}

\AtEveryBibitem{\clearfield{doi}}
\AtEveryBibitem{\clearfield{url}}
\AtEveryBibitem{\clearfield{isbn}}
\AtEveryBibitem{\clearfield{issn}}
\AtEveryBibitem{\clearfield{eprint}}

\DeclareNameAlias{sortname}{last-first}

% Order names: Mustermann, E.
%\DeclareNameAlias{default}{last-first}
% Order names: E. Mustermann
\DeclareNameAlias{author}{first-last}

\renewbibmacro{in:}{}

% Separation between two bibliography items
\setlength\bibitemsep{1.5\itemsep}

\addbibresource{literatur.bib}


%---------------------------------------------------------------------%
% Acronyms

%\input{acronyms.tex}


%---------------------------------------------------------------------%
% Settings

\graphicspath{{./img/}}


%---------------------------------------------------------------------%
% Page styles

\usepackage[]{scrlayer-scrpage}
\automark[subsection]{section}


%---------------------------------------------------------------------%
% Author

\author{Tim Bohne}
\title{Heuristische Lösungsverfahren für Stacking-Probleme mit Transportkosten}
\date{\today}

\newcommand{\ThesisType}{Bachelorarbeit}  %bei Masterarbeit anpassen
\newcommand{\ThesisSupFst}{Prof. Dr. Sigrid Knust}
\newcommand{\ThesisSupSnd}{Tobias Oelschlägel, M. Sc.}

% Darstellung von Theoremen mit normaler Schrift
\theoremstyle{definition}

% Definition von eigenen Umgebungen
\newtheorem{mytheorem}{Theorem}
\newtheorem{mylemma}[mytheorem]{Lemma}
\newtheorem{mydefinition}[mytheorem]{Definition}

%=====================================================================%

\begin{document}

%---------------------------------------------------------------------%
% Frontmatter

\pagestyle{empty}
\pagenumbering{alph}

\input{sections/titlepage.tex}

\cleardoublepage

\renewenvironment{abstract}
{\vspace*{\fill}
  {\bfseries\abstractname}\\[5mm]}
{\vfill}

\begin{abstract}

Stacking-Probleme beschreiben Situationen, in denen es darum geht, eine Menge von Items zulässigen Positionen in Stacks zuzuordnen, sodass bestimmte Nebenbedingungen respektiert werden und ggf. eine Zielfunktion optimiert wird. Sie treten in der Praxis häufig im Umfeld von Lagerhallen und Container-Terminals auf. In dieser Arbeit werden heuristische Lösungsverfahren für verschiedene Varianten von Stacking-Problemen entwickelt, bei denen der Fokus auf der Minimierung der Transportkosten liegt. Dabei werden MIP-Formulierungen zum experimentellen Vergleich genutzt.

\end{abstract}

\selectlanguage{english}
\begin{abstract}

Stacking problems describe situations in which a set of items has to be assigned to feasible
positions in stacks, such that certain constraints are respected and, if necessary, an objective function is optimized.
In practice, such problems for example occur in warehouses and container terminals.
In the present work heuristic approaches are developed for various stacking problems, where the focus is on
minimizing transport costs. MIP formulations are used for experimental comparison.

\end{abstract}
\selectlanguage{ngerman}


\cleardoublepage

\pagenumbering{Roman}

\tableofcontents{}

\cleardoublepage

\pagestyle{scrheadings}
\pagenumbering{arabic}

%---------------------------------------------------------------------%
% Content

\section{Einleitung}
\label{sec:einleitung}

Stacking-Probleme beschreiben Situationen, in welchen es darum geht, eine Menge von Items, häufig Container,
unter Berücksichtigung bestimmter Nebenbedingungen in einer Lagerfläche zu positionieren. Das \textquote{Stacking}
bezieht sich auf die Tatsache, dass das Stapeln der Items essenziell ist, um möglichst viele Items
auf geringer Fläche unterzubringen. In der Praxis treten solche Probleme zum Beispiel im Umfeld von Lagerhallen
und Container-Terminals auf.\newline
Das System des intermodalen Frachttransports durch Container spielt eine zentrale Rolle im internationalen Handel.
Der schnellste und kosteneffektivste Weg, allgemeine Frachten zu verschicken, ist typischerweise der
Containertransport (vgl. \citet{Briskorn2018}).
Mit steigendem Durchsatz gewinnt die effiziente Verwaltung der Container an Bedeutung.
Der Containerumschlag am Hamburger Hafen hat sich beispielsweise zwischen 1990 und 2017
annähernd verdreifacht (vgl. \citet{Port_of_Hamburg}).
Folglich stellt das sinnvolle Ein- und Auslagern der Container eine immer größere Herausforderung dar und
es herrscht ein großer Bedarf, diese Prozesse zu optimieren.

Bisher existieren kaum Veröffentlichungen, welche effiziente Lösungsverfahren für Stacking-Probleme thematisieren.
Insbesondere die effiziente Minimierung der Transportkosten, die beim Verladen der Items von den Fahrzeugen,
mit welchen sie geliefert werden, in die Lagerfläche, entstehen, hat in der Literatur noch keine Beachtung gefunden,
obwohl diese auch aus praktischer Perspektive von Relevanz ist. Ziel dieser Arbeit ist es dementsprechend,
ebensolche heuristische Verfahren für verschiedene Varianten von Stacking-Problemen zu entwickeln, bei denen der Fokus auf
Zulässigkeit und geringer Laufzeit liegt.\newline
Es gibt deutlich mehr Veröffentlichungen zu Unloading-, also Auslagerungs-Szenarien und Szenarien, bei denen simultan Items
ein- und ausgelagert werden, als zu reinen Loading-Problemen, welche Gegenstand dieser Arbeit sind.
Der folgende Literaturüberblick beschränkt sich auf diese wenigen Veröffentlichungen, welche zusätzlich bestimmte Konzepte,
die im Verlauf der Arbeit verwendet oder erweitert werden, etablieren, oder themennahe Problemstellungen behandeln.

\citet{Kim2000} betrachten ein Loading-Problem, bei welchem es darum geht, die Positionierung eintreffender
Container basierend auf ihrem Gewicht vorzunehmen. Es wird ein Ansatz mit dynamischer Programmierung beschrieben,
welcher die Position der Container in der Lagerfläche bestimmt und dabei die Anzahl der erwarteten
Relocation-Operationen\footnote{Umsortierungen innerhalb der Stacks.} minimiert.
Basierend auf den damit generierten Optimallösungen der betrachteten Instanzen wird ein Decision-Tree, welcher Echtzeit-Entscheidungen ermöglicht, entwickelt.
Dementsprechend handelt es sich bei diesem Ansatz durchaus um ein effizientes Lösungsverfahren für Stacking-Probleme.
Es wird mit der Minimierung der Anzahl der erwarteten Relocation-Operationen allerdings eine andere Zielfunktion
als in dieser Arbeit betrachtet.\newline
\citet{Kang2006} beschäftigen sich mit einem ähnlichen Problem und beschreiben einen Simulated-Annealing-Ansatz,
welcher Stacking-Strategien für Container mit ungewissen Gewichtsinformationen bereitstellt.

\pagebreak

Eine weitere themennahe Veröffentlichung liefern \citet{Delgado2012}, welche sich mit einem
Loading-Problem beschäftigen, bei dem ein Containerschiff mit einer Menge an Items beladen wird.
Basierend auf der Stabilität des Schiffs sind für jeden Stack Gewichts- und Höhenbeschränkungen gegeben.
Da einige Container eine Energiequelle benötigen, sind zusätzlich bestimmte Restriktionen bezüglich der
Positionierung gegeben.
Die betrachtete Zielfunktion entspricht der gewichteten Summe vierer Zielfunktionen.
Es werden \textquote{Overstows}\footnote{Container, welche auf einem anderen Container mit früherem Auslagerungszeitpunkt platziert sind.} minimiert und das Platzieren von Containern mit unterschiedlichen Zielhäfen
im selben Stack, die Eröffnung neuer Stacks sowie das Platzieren eines Containers,
welcher keine Energiequelle benötigt, an einer Position, an welcher eine solche vorliegt, vermieden.
Durch eine Reduktion vom \textsc{BinPacking}-Problem wird gezeigt,
dass die Minimierung der Anzahl verwendeter Stacks NP-schwer ist.
Die Restriktionen bezüglich der Positionierung bestimmter Items sind auch Teil der Nebenbedingungen,
welche in dieser Arbeit betrachtet werden.

Im Zusammenhang mit Bahn-Terminals betrachtet \citet{Jaehn2013} ein Loading-Problem,
bei welchem Items von einer Anzahl von Zügen, welche gleichzeitig eintreffen,
in eine Lagerfläche verladen werden, welche aus parallelen \textquote{Lanes} mit limitierter Kapazität besteht.
Es wird zunächst die NP-Schwere zweier Modelle gezeigt, woraufhin heuristische Algorithmen vorgestellt werden,
welche mit realen Daten getestet werden. Eine der dort entwickelten Heuristiken kommt sogar in der Praxis
in einem Hafen zum Einsatz. Auch wenn es sich prinzipiell um eine der Problemstellung in dieser
Arbeit nicht ferne Ausgangssituation handelt, gibt es doch einige grundlegende Unterschiede.
Unter anderem stellt das Stapeln von Containern im dort betrachteten Szenario eher die Ausnahme dar.

\citet{Bruns2015} haben erste Ergebnisse zur Komplexität verschiedener Stacking-Probleme veröffentlicht.
Daraus stammt unter anderem die Erkenntnis, dass eine der in dieser Arbeit betrachteten Problemstellungen stark NP-vollständig ist.
Des Weiteren wurden Polynomialzeit-Algorithmen für einige Stacking-Probleme vorgestellt,
welche zum Teil in den im Verlauf dieser Arbeit entwickelten Heuristiken Anwendung finden.\newline
Gleichermaßen Verwendung in dieser Arbeit finden die von \citet{Le2016} entwickelten
MIP-Formulierungen\footnote{MIP: Mixed-Integer-Programming.} zur Lösung von Loading-Problemen, bei welchen es darum geht,
die Anzahl der verwendeten Stacks zu minimieren. Diese ermöglichen in angepasster Form einen experimentellen Vergleich mit
den entwickelten Heuristiken in Bezug auf die Lösungsqualität.

Diese Beispiele stehen exemplarisch dafür, dass es sich bei Storage-Loading- bzw. Stacking-Problemen um
ein aktives Forschungsgebiet handelt, welches auch aus praktischer Perspektive von großer Relevanz ist.
Außerdem verdeutlichen sie, dass der Fokus bisher in der Regel auf der Optimierung anderer Zielfunktionen lag.

Zunächst werden in Kapitel \ref{sec:stacking_problems} Stacking-Probleme im Allgemeinen
eingeführt und anschließend formal definiert.
Da­r­auf­fol­gend geht es in Kapitel \ref{sec:test_data} um die Testinstanzen, welche generiert werden,
um die entwickelten Heuristiken testen und miteinander vergleichen zu können.
Ebenfalls dem experimentellen Vergleich dienen die in Kapitel \ref{sec:mip_formulations} eingeführten MIP-Formulierungen,
welche durch Ermittlung der optimalen Zielfunktionswerte eine Beurteilung der Lösungsqualität der Heuristiken ermöglichen.
In Kapitel \ref{sec:constructive_heuristics} werden die entwickelten konstruktiven Heuristiken vorgestellt und
die Ergebnisse der experimentellen Vergleiche präsentiert. Anschließend wird in Kapitel \ref{sec:post_optimization} ein Verbesserungsverfahren entwickelt, welches eine von den Heuristiken aus Kapitel \ref{sec:constructive_heuristics} generierte Lösung als Eingabe erhält.
Dieses Verfahren hat zum Ziel, die Initiallösung basierend auf einer lokalen Suche
zu verbessern. Abschließend wird in Kapitel \ref{sec:conclusion} ein Fazit formuliert.


\pagebreak

\section{Stacking-Probleme}
\label{sec:stacking_problems}

Stacking-Probleme treten in der Praxis zum Beispiel im Umfeld von Lagerhallen und Container-Terminals auf.
Ankommende \textit{Items}, häufig Container, müssen sogenannten \textit{Stacks} zugeordnet werden, sodass bestimmte Nebenbedingungen respektiert werden.
Diese Nebenbedingungen sorgen zum Beispiel dafür, dass nicht jedes \textit{Item} auf jedem anderen \textit{Item} platziert werden
darf. Es wird dabei angenommen, dass die im Folgenden als \textit{Storage-Area} bezeichnete Lagerfläche in festen
\textit{Stacks} organisiert ist, die eine limitierte gemeinsame Höhe besitzen, die im Folgenden als \textit{Stack}-Kapazität bezeichnet wird.\newline
Es kann in der Regel nur auf das oberste \textit{Item} eines \textit{Stacks} zugegriffen werden, d.h. der \textit{Item}-Zugriff erfolgt
nach dem \textquote{Last-In–First-Out}-Prinzip. Ein Zugriff auf \textit{Items} darunter erfordert eine \textit{Relocation}-Operation, die
zeit- und energieaufwendig ist.\newline

In dieser Arbeit wird nur der \textit{Loading}-Prozess betrachtet, währenddem keine \textit{Item}-Entnahme stattfindet.
Die \textit{Items} werden in die \textit{Storage-Area} geladen, die aus \textit{Stacks} mit einer festen Position besteht, d.h.
man kann die \textit{Stacks} nicht positionieren, sondern lediglich den \textit{Items} eine Position in einem \textit{Stack} zuweisen.\newline
Der \textit{Relocation}-Prozess wird typischerweise durch Kräne durchgeführt, die den \textit{Stacks} eine Maximalhöhe auferlegen.

Neben der Tatsache, dass nicht jedes \textit{Item} auf jedem anderen \textit{Item} platziert werden darf, gilt auch, dass es \textit{Items} gibt, die auf bestimmte Positionen beschränkt sind, weil sie zum Beispiel eine Energiequelle benötigen.
Abbildung \ref{fig:storage_area} zeigt den Aufbau der \textit{Storage-Area} bestehend aus $m$ fixierten
\textit{Stacks}, die jeweils $b$ \textit{Level} enthalten. Jedes \textit{Level} innerhalb eines \textit{Stacks} entspricht dabei einer Position, der ein \textit{Item} zugeordnet werden kann.

\begin{figure}[H]
\centering
\resizebox{0.5\textwidth}{!}{
\begin{tabular}{|c|c|c|c|c|c|}
\cline{1-5}
$\boldsymbol{S_1}$ & $\boldsymbol{S_2}$ & $\boldsymbol{S_3}$ & ... & $\boldsymbol{S_m}$\\ \cline{1-5}
\end{tabular}}
\caption*{\textsc{von oben betrachtet}}
\end{figure}
\begin{figure}[H]
\centering
\resizebox{0.5\textwidth}{!}{
\begin{tabular}{c|c|c|c|c|c|}
\cline{2-6}
$\boldsymbol{L_b}$ & $pos_{1, b}$ & $pos_{2, b}$ & $pos_{3, b}$ & ... & $pos_{m, b}$ \\ \cline{2-6}
... & ... & ... & ... & ... & ...\\ \cline{2-6}
$\boldsymbol{L_2}$ & $pos_{1, 2}$ & $pos_{2, 2}$ & $pos_{3, 2}$ & ... & $pos_{m, 2}$\\ \cline{2-6}
$\boldsymbol{L_1}$ & $pos_{1, 1}$ & $pos_{2, 1}$ & $pos_{3, 1}$ & ... & $pos_{m, 1}$\\ \cline{2-6}
\multicolumn{1}{c}{} & \multicolumn{1}{c}{$\boldsymbol{S_1}$} & \multicolumn{1}{c}{$\boldsymbol{S_2}$}
& \multicolumn{1}{c}{$\boldsymbol{S_3}$} & \multicolumn{1}{c}{...} & \multicolumn{1}{c}{$\boldsymbol{S_m}$} \\
\end{tabular}}
\caption*{\textsc{von der Seite betrachtet}}

\caption{\textsc{Aufbau der \textit{Storage-Area}}}
\label{fig:storage_area}
\end{figure}

Das Ziel von Stacking-Problemen ist es, jedes ankommende \textit{Item} einer zulässigen Position in einem
\textit{Stack} zuzuordnen, sodass eine gegebene Zielfunktion optimiert wird. Es gibt eine ganze Reihe praktisch relevanter Zielfunktionen,
in dieser Arbeit geht es um die Minimieren der Transportkosten, die bei der Verladung der \textit{Items} von ihrer Originalposition zur zugewiesenen Stackposition entstehen.

%%%%%%%%%%%%%%%%%%%%%%%%%%%%%%%%%%%%%%%%%%%%%%%%%%%%%%%%%%%%%%%%%%%%%%%%%%%%%%%%%%%%%%%%%%%%%%%%%%%%
\pagebreak
%%%%%%%%%%%%%%%%%%%%%%%%%%%%%%%%%%%%%%%%%%%%%%%%%%%%%%%%%%%%%%%%%%%%%%%%%%%%%%%%%%%%%%%%%%%%%%%%%%%%

\section{Formale Definition}
\label{sec:formal_definition}

Die \textit{Storage-Area} besteht aus \textit{Stacks}, die in einer Reihe angeordnet sind (vgl. Abb. \ref{fig:storage_area}). Die x-Koordinate spezifiziert jeweils den \textit{Stack} und die y-Koordinate den \textit{Level} innerhalb des jeweiligen \textit{Stacks}. Es handelt sich also um eine Reihe von $m$ \textit{Stacks} mit jeweils $b$ Positionen pro \textit{Stack}. Abb. \ref{fig:parameters} zeigt eine Liste der
Parameter, die im weiteren Verlauf der Arbeit immer wieder verwendet werden, um Stacking-Probleme zu definieren.
\begin{figure}[H]
\centering
\resizebox{0.5\textwidth}{!}{
\begin{tabular}{ | l | l |}
    \hline
    \textbf{Parameter} & \textbf{Semantik} \\ \hline
    $n$ & Anzahl der Items \\ \hline
    $m$ & Anzahl der Stacks \\ \hline
    $b$ & Stack Kapazität \\ \hline
    $I$ & Menge der Items $ I := \{1, 2, ..., n\}$ \\ \hline
\end{tabular}}
\caption{\textsc{Zur Definition verwendete Parameter}}
\label{fig:parameters}
\end{figure}
In der Regel gilt $m < n$, d.h. es müssen \textit{Items} gestapelt werden.
Außerdem muss die Annahme $n \leq bm$ gelten, denn sonst ist die Instanz des Problems unzulässig,
weil es mehr \textit{Items} als Positionen gibt.\newline
Für jedes \textit{Item} $i \in I$ ist dessen Originalposition $O_i$ in x- und y-Koordinaten gegeben.

\textit{Items}, die in einem \textit{Stack} platziert werden, sind durch ein Tupel $(i_k, ..., i_1)$ definiert, wobei
$i_\lambda$ das \textit{Item} auf \textit{Level} $\lambda$ beschreibt. $\lambda = 1$ beschreibt den \textit{Ground-Level}.
Instanzen können Stacking-Restriktionen $s_{ij}$ enthalten, die angeben, ob ein \textit{Item} $i$ direkt auf einem \textit{Item} $j$ platziert
werden darf. Dies ist der Fall, wenn $s_{ij} = 1$ gilt.
Ein Tupel ist zulässig, wenn $k \leq b$ und $s_{i_{\lambda + 1}, i_\lambda} = 1 \quad \forall \quad \lambda = 1, ..., k - 1$.
D.h. ein Tupel ist dann zulässig, wenn die \textit{Stack}-Kapazität eingehalten wird und alle \textit{Items}, die aufeinander gestapelt sind,
nicht den Stacking-Restriktionen widersprechen, die in Kapitel \ref{stacking_restrictions} erläutert werden.

\textbf{Formulierung des Problems}\newline
Gegeben sei eine Menge $I = \{1, ..., n\}$ von \textit{Items} und eine Menge $Q = \{1, ..., m\}$ von \textit{Stacks} mit einer \textit{Stack}-Kapazität von $b$. Außerdem seien \textit{Stacking Constraints} $s_{ij}$ und \textit{Placement Constraints} $t_{iq}$ gegeben.
Das Ziel ist nun, jedes \textit{Item} $i \in I$ genau einem \textit{Stack} $q \in Q$ zuzuweisen, wobei die \textit{Stacking Constraints} $s_{ij}$,
die Placement Constraints $t_{iq}$ und die Stack-Kapazität $b$ respektiert werden und gegebenenfalls eine Zielfunktion optimiert wird.

%%%%%%%%%%%%%%%%%%%%%%%%%%%%%%%%%%%%%%%%%%%%%%%%%%%%%%%%%%%%%%%%%%%%%%%%%%%%%%%%%%%%%%%%%%%%%%%%%%%%
\pagebreak
%%%%%%%%%%%%%%%%%%%%%%%%%%%%%%%%%%%%%%%%%%%%%%%%%%%%%%%%%%%%%%%%%%%%%%%%%%%%%%%%%%%%%%%%%%%%%%%%%%%%

\section{Das Zulässigkeitsproblem}
\label{sec:decision_problem}

Die simpelste Variante des Storage Loading Problems ist ein Zulässigkeitsproblem, welches die Frage stellt, ob sämtliche Items der Storage Area zugewiesen werden können, sodass alle Nebenbedingungen respektiert werden.
Diese Nebenbedingungen beziehen sich z.B. auf die Stack Kapazität, die Stacking Constraints, ein Gewichtslimit pro Stack oder auch
Restriktionen bezüglich der Position.\newline
Sollte dies möglich sein, so ist häufig das Ziel, jedes Item einer zulässigen Position zuzuweisen.
Diese Position zeichnet sich durch einen Stack-Index und einen Level innerhalb des Stacks aus.
Dabei kann z.B. eine der in Abb. \ref{fig:objective_functions} dargestellten Zielfunktionen optimiert werden, was jedoch bereits über
das Zulässigkeitsproblem hinausgeht.

%%%%%%%%%%%%%%%%%%%%%%%%%%%%%%%%%%%%%%%%%%%%%%%%%%%%%%%%%%%%%%%%%%%%%%%%%%%%%%%%%%%%%%%%%%%%%%%%%%%%
\pagebreak
%%%%%%%%%%%%%%%%%%%%%%%%%%%%%%%%%%%%%%%%%%%%%%%%%%%%%%%%%%%%%%%%%%%%%%%%%%%%%%%%%%%%%%%%%%%%%%%%%%%%

\section{Transportkosten}
\label{sec:transport_costs}
Die Transportkosten beziehen sich auf die Kosten, die bei der Verladung der \textit{Items} in die \textit{Storage-Area} entstehen
und sind praktisch zum Beispiel durch Kranbetriebskosten, Wartezeiten oder auch Arbeitszeiten motiviert.\newline
Jeder Stack $q$ besitzt eine fixierte Position $F_q$ in der Storage-Area. Außerdem besitzt jedes Item $i$ eine gegebene
Originalposition $O_i$ auf dem Fahrzeug, mit dem es geliefert wird. Die Kosten ergeben sich aus der Manhattan-Distanz zwischen
Item- und zugewiesener Stack-Position, da diese der Kranbewegung entspricht. In Abb. \ref{fig:costs} ist dieser Sachverhalt visualisiert.
\begin{figure}[H]
\includegraphics[width=\textwidth]{img/costs.png}
\caption{Beispielszenario}
\label{fig:costs}
\end{figure}
Die Transportkosten werden in einer Matrix $C = (c_{iq})_{n \times m}$ kodiert, \newline wobei $c_{iq} \coloneqq d_{man}(O_i, F_q)$.

%%%%%%%%%%%%%%%%%%%%%%%%%%%%%%%%%%%%%%%%%%%%%%%%%%%%%%%%%%%%%%%%%%%%%%%%%%%%%%%%%%%%%%%%%%%%%%%%%%%%
\pagebreak
%%%%%%%%%%%%%%%%%%%%%%%%%%%%%%%%%%%%%%%%%%%%%%%%%%%%%%%%%%%%%%%%%%%%%%%%%%%%%%%%%%%%%%%%%%%%%%%%%%%%

\section{Nebenbedingungen}
\label{sec:constraints}

\subsection{Stacking Restriktionen}
\label{sec:stacking_restrictions}

Es gibt in der Praxis vielfältige Gründe dafür, dass nicht jedes \textit{Item} auf jedes andere \textit{Item} gestapelt werden darf.
Aus diesen resultieren bestimmte Restriktionen, die z.B. wie folgt lauten:
\begin{itemize}
  \item schwerere \textit{Items} dürfen nicht auf leichteren \textit{Items} platziert werden
  \item größere \textit{Items} dürfen nicht auf kleineren \textit{Items} platziert werden
  \item \textit{Items} bestimmter Materialien oder Zielorte dürfen nicht aufeinander gestapelt werden
\end{itemize}

All jene Restriktionen, die im Folgenden stets als \textit{Stacking Constraints} bezeichnet werden, werden in einer Binärmatrix $S = (s_{ij})_{n \times n}$ kodiert, wobei:
\[
    s_{ij} =
\begin{cases}
    1, & \text{wenn $i$ direkt auf $j$ gestapelt werden darf }\\
    0, & \text{sonst}
\end{cases}
\]
Diese Matrix lässt sich, wie in Abb. \ref{fig:matrix_to_graph} beispielhaft dargestellt,
als gerichteter Graph mit $n$ Knoten für alle $i \in I$ und Kanten $i \rightarrow j$ für alle $s_{ij} = 1$ repräsentieren.
\begin{figure}[H]
  \begin{subfigure}[b]{0.5\textwidth}
  \centering
    $S =$
    $\left(
    \begin{array}{rrrr}
    0 & 1 & 1 & 0 \\
    1 & 0 & 1 & 0 \\
    1 & 0 & 0 & 0 \\
    1 & 1 & 1 & 0 \\
    \end{array} \right) $
    \caption{\textsc{Constraint Matrix}}
    \label{fig:constraint_matrix}
  \end{subfigure}
  \hfill
  \begin{subfigure}[b]{0.5\textwidth}
  \centering
    \begin{tikzpicture}[->, scale=0.50, transform shape, node distance=2.5cm]
    \node[state] (A) {1};
    \node[state] (B) [right of=A] {2};
    \node[state] (C) [below of=A] {3};
    \node[state] (D) [right of=C] {4};

    \path (A) edge node {} (B)
          (A) edge node {} (C)

          (B) edge [bend right] node {} (A)
          (B) edge [bend left=10] node {} (C)

          (C) edge [bend left] node {} (A)

          (D) edge [bend left=10] node {} (A)
          (D) edge node {} (B)
          (D) edge node {} (C);
  \end{tikzpicture}
    \caption{\textsc{Resultierender Graph}}
    \label{fig:resulting_graph}
  \end{subfigure}
  \caption{\textsc{Stacking Constraints interpretiert als Matrix bzw. Graph.}}
  \label{fig:matrix_to_graph}
\end{figure}
\textit{Stacking Constraints} können transitiv sein. Ist dies der Fall, so gilt folgendes:\newline
Wenn \textit{Item} $i$ auf \textit{Item} $j$ und \textit{Item} $j$ auf \textit{Item} $h$ gestapelt werden darf,
so darf auch \textit{Item} $i$ auf \textit{Item} $h$ gestapelt werden.\newline
Restriktionen bezüglich des Gewichts und der Länge haben die besondere Eigenschaft, dass alle Items vergleichbar sind, d.h. für alle $i \neq j  \quad \text{gilt} \quad s_{ij} = 1 \quad \text{oder} \quad s_{ji} = 1$, was bedeutet, dass diese Restriktionen eine totale Ordnung auf allen \textit{Items} definieren.

%%%%%%%%%%%%%%%%%%%%%%%%%%%%%%%%%%%%%%%%%%%%%%%%%%%%%%%%%%%%%%%%%%%%%%%%%%%%%%%%%%%%%%%%%%%%%%%%%%%%
\pagebreak
%%%%%%%%%%%%%%%%%%%%%%%%%%%%%%%%%%%%%%%%%%%%%%%%%%%%%%%%%%%%%%%%%%%%%%%%%%%%%%%%%%%%%%%%%%%%%%%%%%%%

\subsection{Placement Restriktionen}
\label{sec:placement_restrictions}

Die im Folgenden als \textit{Placement Constraints} bezeichneten Einschränkungen bezüglich der Position eines Items resultieren z.B.
aus der Länge oder dem Gewicht der Items und den Eigenschaften der Stacks oder auch durch spezielle Anforderungen der Items wie
Kühlcontainer, die eine Energiequelle benötigen. Tendenziell sind dabei mehr Item-Stack-Zuweisungen erlaubt als verboten.
Diese Placement-Constraints werden nun allerdings nicht direkt implementiert, sondern indirekt über hohe Kostenwerte.\newline

Zunächst werden die Placement-Constraints in einer Binärmatrix $\boldsymbol{T} = (\boldsymbol{t_{iq}})_{n \times m}$ kodiert, wobei:\newline
$
    \boldsymbol{t_{iq}} =
\begin{cases}
    1, & \text{wenn Item $\boldsymbol{i}$ in Stack $\boldsymbol{q}$ platziert werden darf }\\
    0, & \text{sonst}
\end{cases}
$
\\[20pt]

Diese Matrix wird jedoch nicht direkt verwendet, um die Restriktionen zu implementieren. Stattdessen wird
die Transportkosten-Matrix $C$ nun wie folgt erweitert.

Für die Transportkosten-Matrix $\boldsymbol{C} = (\boldsymbol{c_{iq}})_{n \times m}$ gilt:\newline
$
    \boldsymbol{c_{iq}} =
\begin{cases}
    d_{man}(\boldsymbol{O_i}, \boldsymbol{F_q}), & \text{wenn $\boldsymbol{t_{iq}} = 1$}\\
    \infty, & \text{sonst}
\end{cases}
$

%%%%%%%%%%%%%%%%%%%%%%%%%%%%%%%%%%%%%%%%%%%%%%%%%%%%%%%%%%%%%%%%%%%%%%%%%%%%%%%%%%%%%%%%%%%%%%%%%%%%
\pagebreak
%%%%%%%%%%%%%%%%%%%%%%%%%%%%%%%%%%%%%%%%%%%%%%%%%%%%%%%%%%%%%%%%%%%%%%%%%%%%%%%%%%%%%%%%%%%%%%%%%%%%

\section{Problemstellungen}
\label{sec:problem_settings}

\subsection{Komplexität}
\label{sec:complexity}

In diesem Kapitel werden konkret die Problemstellungen eingeführt, die im Folgenden heuristisch gelöst werden.
Es werden die Stack-Kapazitäten $b=2, b=3 $ und $ b=4$ betrachtet, die besonders in Szenarien aus der Praxis, in denen
Container gestapelt werden, von Interesse sind.

Wie \cite{Bruns2015} zeigte, handelt es sich bei dem Zulässigkeitsproblem mit $b=3$ und gegebenen $s_{ij}$ um
ein NP-vollständiges Problem. Erst kürzlich wurde außerdem gezeigt, dass auch $b=2$ mit gegebenen $s_{ij}$ und $t_{iq}$
NP-vollständig ist. Somit sind auch die entsprechenden Stacking-Probleme, die zusätzlich die Transportkosten minimieren, NP-vollständig.

\subsection{Instanzgrößen}
\label{sec:instance_sizes}

Die betrachteten Testinstanzen werden in drei Kategorien unterteilt:
\begin{itemize}
  \item klein (\textbf{s}) ($\leq 100$ Items)
  \item mittel (\textbf{m}) ($\approx 300$ Items)
  \item groß (\textbf{l}) ($\approx 500$ Items)\newline
\end{itemize}
Dies scheint auch in der Literatur der Rahmen zu sein, in dem sich solche Stacking-Probleme typischerweise bewegen.

\subsection{Ziel}
\label{sec:objective}

Sämtliche Items sollen möglichst günstig eingelagert werden, d.h. die Zielfunktion entspricht der Minimierung der Transportkosten.
Dabei haben Zulässigkeit und geringe Laufzeit allerdings Priorität.


%%%%%%%%%%%%%%%%%%%%%%%%%%%%%%%%%%%%%%%%%%%%%%%%%%%%%%%%%%%%%%%%%%%%%%%%%%%%%%%%%%%%%%%%%%%%%%%%%%%%
\pagebreak
%%%%%%%%%%%%%%%%%%%%%%%%%%%%%%%%%%%%%%%%%%%%%%%%%%%%%%%%%%%%%%%%%%%%%%%%%%%%%%%%%%%%%%%%%%%%%%%%%%%%

\section{Testdaten-Generierung}
\label{sec:test_data}

In diesem Kapitel geht es um die durchaus interessante Frage danach, wie realistische Test-Instanzen aussehen können.
Zunächst einmal werden pro Konfiguration 20 Instanzen verwendet um eine Vergleichbarkeit zu ermöglichen.
Die Anzahl der Items $n$ sowie die Stack-Kapazität $b$ wird stets spezifiziert und in der Berechnung der Anzahl
der zur Verfügung stehenden Stacks $m$ verwendet: $m$ = $\ceil{n / b}$ + $20 \%$.\newline
Bei einer Stackanzahl von $m$ = $\ceil{n / b}$ gäbe es überhaupt keinen Spielraum bei der Itemplatzierung. Ein Spielraum von $20 \%$
ist durchaus gering und könnte in weiteren Betrachtungen erhöht werden.

\textbf{Stacking-Constraint-Generierung}\newline

Es wurden zwei Varianten der Stacking-Constraint-Generierung implementiert, die im Folgenden erläutert werden.

\textbf{Variante 1}\newline
Die Stacking-Constraint Matrix $S$ wird anhand einer definierten Wahrscheinlichkeit mit $1$- bzw. $0$-Einträgen gefüllt.
Anschließend werden die 1-Einträge, die aus Transitivität folgen, ergänzt.

\textbf{Variante 2}\newline
Für alle $n$ Items wird eine zufällige Länge und Breite generiert. Item $i$ kann auf Item $j$ platziert werden,
wenn für dessen Länge $l_i$ und Breite $w_i$ gilt $l_i \leq l_j$ und $w_i \leq w_j$.
Items $i, j$ sind in beide Richtungen stackbar, wenn $l_i = l_j$ und $w_i = w_j$.

Variante 2 scheint realitätsnäher zu sein und wird deshalb zunächst bevorzugt. Aufgrund der Tatsache, dass Variante 2 jedoch stets
zu einer 1-Quote von ca. 25\% führt und diese mit Variante 1 sehr gut konfigurierbar ist, wird Variante 1 jedoch noch nicht vollständig verworfen.
(TODO: Grund ergänzen...)

Bei den Placement-Constraints sind in den betrachteten Tests 70\% der Item-Stack-Zuweisungen erlaubt.
Zur Transportkostenberechnung wird wie in Abschnitt \ref{sec:transport_costs} erläutert, die Manhattan-Metrik verwendet. Dazu sind
entsprechend die Item- und Stackpositionen als x- und y-Koordinaten gegeben. Auch die entsprechenden Längen und Breiten sowie die Distanz zwischen Storage-Area und den Fahrzeug-Lanes ist jeweils gegeben.

%%%%%%%%%%%%%%%%%%%%%%%%%%%%%%%%%%%%%%%%%%%%%%%%%%%%%%%%%%%%%%%%%%%%%%%%%%%%%%%%%%%%%%%%%%%%%%%%%%%%
\pagebreak
%%%%%%%%%%%%%%%%%%%%%%%%%%%%%%%%%%%%%%%%%%%%%%%%%%%%%%%%%%%%%%%%%%%%%%%%%%%%%%%%%%%%%%%%%%%%%%%%%%%%

\section{MIP-Formulierungen}
\label{sec:mip_formulations}

In diesem Abschnitt werden die MIP-Formulierungen eingeführt, die zum experimentellen Vergleich genutzt werden.
Es geht darum, mit den MIP-Formulierungen die optimalen Zielfunktionswerte der Instanzen zu ermitteln, um dann
zu ermitteln, wie weit die Heuristiken davon abweichen. Des Weiteren geht es auch um einen Vergleich der MIP-Formulierungen
untereinander.

\subsection{Bin-Packing-Formulierung}
\label{sec:bin_packing_formulation}

\begin{gather}
\boldsymbol{min} \quad \sum_{i \in I} \sum_{q \in Q} c_{iq} x_{iq} \\
\thinspace\thinspace \boldsymbol{s.t.} \thinspace\quad\thinspace\thinspace\thinspace\quad\thinspace \quad \sum_{q \in Q} x_{iq} = 1 \quad\quad\quad\quad\quad\quad\quad \forall i \in I \quad\quad\quad\quad\thinspace\thinspace \\
\quad \sum_{i \in I} x_{iq} \leq b \quad\quad\quad\quad\quad\quad\quad \forall q \in Q \\
\thinspace\thinspace\quad\quad x_{iq} + x_{jq} \leq 1 \quad\quad\thinspace\thinspace\quad\quad\quad\quad \forall \{i, j\} \notin A \\
\thinspace\thinspace\quad\quad\quad x_{iq} \in \{0, 1\} \thinspace\thinspace\thinspace\thinspace\thinspace\thinspace\quad\quad\quad\quad\quad\quad \forall i \in I, q \in Q \thinspace
\end{gather}

\subsection{3-Index-Formulierung}
\label{sec:three_idx_formulation}

\begin{gather}
\boldsymbol{min} \quad \sum_{i \in I} \sum_{q \in Q} \sum_{l \in L} c_{iq} x_{iql} \\
\boldsymbol{s.t.} \quad \sum_{q \in Q} \sum_{l \in L} x_{iql} = 1 \quad\thinspace\thinspace\quad\quad\quad\quad \forall i \in I \thinspace\thinspace\quad\quad\quad\quad\quad\quad\quad\quad\thinspace\thinspace \\
\sum_{i \in I} x_{iql} \leq 1 \thinspace\quad\quad\quad\quad\quad\quad\quad \forall q \in Q, l \in L \thinspace\thinspace\thinspace\thinspace\quad\quad\quad\thinspace \\
\sum_{j \in I | i \rightarrow j} x_{jq(l-1)} - x_{iql} \geq 0 \thinspace\quad\quad \forall i \in I, q \in Q, l \in L \textbackslash \{1\} \\
\quad\quad x_{iql} \in \{0, 1\} \quad\quad\quad\quad\quad\quad\quad \forall i \in I, q \in Q, l \in L \quad\thinspace\thinspace\thinspace\thinspace\thinspace
\end{gather}


%%%%%%%%%%%%%%%%%%%%%%%%%%%%%%%%%%%%%%%%%%%%%%%%%%%%%%%%%%%%%%%%%%%%%%%%%%%%%%%%%%%%%%%%%%%%%%%%%%%%
\pagebreak
%%%%%%%%%%%%%%%%%%%%%%%%%%%%%%%%%%%%%%%%%%%%%%%%%%%%%%%%%%%%%%%%%%%%%%%%%%%%%%%%%%%%%%%%%%%%%%%%%%%%

\section{Konstruktive Heuristiken}
\label{sec:constructive_heuristics}

In diesem Abschnitt werden die entwickelten konstruktiven Heuristiken vorgestellt.
Zunächst werden drei wichtige Konzepte vorgestellt, in \ref{sec:digression_mcm} das Maximum-Cardinality-Matching,
in \ref{sec:digression_mwpm} das Minimum-Weight-Perfect-Matching und in \ref{sec:digression_bipartite_graph} ein
bipartiter Graph.

\subsection{Exkurs: Maximum-Cardinality-Matching (MCM)}
\label{sec:digression_mcm}

Gegeben sei ein ungerichteter Graph $G = (V, E)$. Eine Menge $M \subseteq E$ heißt Matching,
wenn keine zwei Kanten aus $M$ einen Knoten gemeinsam haben. Falls $M$ als Menge eine maximale Kardinalität unter
allen Matchings von $G$ hat, wird dies als Maximum-Cardinality-Matching bezeichnet. \cite{WikiMatching}
Es kann mehrere unterschiedliche MCMs in einem Graphen geben.
In Abb. \ref{fig:mcm_examples} werden beispielhaft zwei MCMs dargestellt.
\begin{figure}[H]
  \begin{subfigure}[b]{0.4\textwidth}
  \centering
    \begin{tikzpicture}[scale=0.45, transform shape, node distance=2cm]
    \node[state] (A) {};
    \node[state] (B) [above right of=A] {};
    \node[state] (C) [below left of=A] {};
    \node[state] (D) [below right of=C] {};
    \node[state] (F) [below right of=B] {};
    \node[state] (E) [below of=F] {};

    \path (A) edge [red] node {} (B)
          (A) edge node {} (C)
          (A) edge node {} (D)
          (A) edge node {} (E)
          (A) edge node {} (F)
          (C) edge [red] node {} (D);
  \end{tikzpicture}
  \caption{\textsc{$|MCM| = 2$}}
  \label{fig:mcm1}
  \end{subfigure}
  \hfill
  \begin{subfigure}[b]{0.4\textwidth}
  \centering
    \begin{tikzpicture}[scale=0.45, transform shape, node distance=2cm]
    \node[state] (A) {};
    \node[state] (B) [right of=A] {};
    \node[state] (C) [right of=B] {};
    \node[state] (D) [below left of=A] {};
    \node[state] (E) [below right of=D] {};
    \node[state] (F) [right of=E] {};

    \path (A) edge node {} (B)
          (B) edge [red] node {} (C)
          (B) edge node {} (F)
          (A) edge node {} (E)
          (E) edge [red] node {} (F)
          (A) edge [red] node {} (D)
          (D) edge node {} (E);
  \end{tikzpicture}
    \caption{\textsc{$|MCM| = 3$}}
    \label{fig:mcm_2}
  \end{subfigure}
  \caption{\textsc{Zwei Beispiele für MCMs (dargestellt in rot).}}
  \label{fig:mcm_examples}
\end{figure}

\subsection{Exkurs: Minimum-Weight-Perfect-Matching}
\label{sec:digression_mwpm}

Zunächst einmal ist es wichtig, zu erwähnen, dass ein Perfect-Matching nur für Graphen mit einer geraden Anzahl an Knoten möglich ist.
Ein Perfect-Matching ist ein Matching, welches sämtliche Knoten des Graphen matcht.
D.h. jeder Knoten des Graphen ist inzident zu genau einer Kante des Matchings (vgl. Abb. \ref{fig:perfect_matching}).
Ein Minimum-Weight-Perfect-Matching ist nun ein günstigstes Perfect-Matching basierend auf den Kantenkosten.
\begin{figure}[H]
\centering
\begin{tikzpicture}[-, scale=0.40, transform shape, node distance=3cm]
    \node[state] (A) {};
    \node[state] (B) [above right of=A] {};
    \node[state] (C) [below right of=A] {};
    \node[state] (D) [right of=B] {};
    \node[state] (E) [right of=D] {};
    \node[state] (F) [right of=C] {};

    \path (A) edge [red] node {} (B)
          (A) edge node {} (C)
          (B) edge node {} (D)
          (B) edge node {} (C)
          (D) edge [red] node {} (E)
          (D) edge node {} (F)
          (C) edge [red] node {} (F);
\end{tikzpicture}
\caption{\textsc{Beispiel für ein Perfect-Matching (rot).}}
\label{fig:perfect_matching}
\end{figure}

\subsection{Exkurs: Bipartiter Graph}
\label{sec:digression_bipartite_graph}

Ein Graph $G = (V, E)$ heißt bipartit (vgl. Abb. \ref{fig:bipartite_graph}), wenn die Knotenmenge in zwei disjunkte Teilmengen zerfällt
$(V = S \cup T$ mit $S \cap T = \emptyset$), sodass jede Kante einen Knoten aus $S$ mit einem Knoten aus $T$ verbindet. \cite{HochschuleDarmstadt}\newline
Es lassen sich folgende Äquivalenzen festhalten:
\begin{itemize}
  \item $G$ bipartit $\iff$ $G$ 2-färbbar
  \item $G$ bipartit $\iff$ keine Kreise ungerader Länge in $G$
\end{itemize}

\begin{figure}[H]
\centering
\begin{tikzpicture}[scale=0.50, transform shape, node distance=3cm]
  \node[state] [green] (A) {};
  \node[state] [red] (B) [above right of=A] {};
  \node[state] [green] (C) [below right of=B] {};
  \node[state] [red] (D) [below of=B] {};
  \node[state] [red] (E) [below of=A] {};
  \node[state] [green] (F) [right of=E] {};

  \path (A) edge node {} (B)
        (B) edge node {} (C)
        (C) edge node {} (D)
        (A) edge node {} (D)
        (A) edge node {} (E)
        (E) edge node {} (F)
        (D) edge node {} (F);

\end{tikzpicture}
\caption{\textsc{Bipartiter Graph.}}
\label{fig:bipartite_graph}
\end{figure}

\subsection{Konstruktive Heuristik ($b = 2$)}
\label{sec:two_cap_heuristic}

\begin{itemize}
  \item Stacking-Constraint-Graph generieren
  \item \textsc{\textbf{MCM}} berechnen, Kanten als Item-Paare interpretieren
  \item Bipartiten Graph generieren:
  \begin{enumerate}
    \item Items (Item-Paare, Unmatched-Items)
    \item Stacks
  \end{enumerate}
  \item \textsc{\textbf{MWPM}} berechnen, Kanten als Stackzuweisungen interpretieren
  \item Ggf. Reihenfolge der Items innerhalb der Stacks anpassen
\end{itemize}

\textbf{Beispiel}
$\boldsymbol{I} := \{0, 1, 2, 3, 4, 5, 6\} \quad\quad\quad \boldsymbol{b} := 2 \quad\quad\quad \boldsymbol{m} := 4$
\begin{figure}[H]
    \centering
    \begin{tikzpicture}[scale=0.8, transform shape, node distance=3cm]
        \node[state] (G) [thick] {$\boldsymbol{6}$};
        \node[state] (B) [thick, above left of=G] {$\boldsymbol{1}$};
        \node[state] (C) [thick, above right of=G] {$\boldsymbol{2}$};
        \node[state] (E) [thick, below right of=G] {$\boldsymbol{4}$};
        \node[state] (F) [thick, below left of=G] {$\boldsymbol{5}$};
        \node[state] (A) [thick, below left of=B] {$\boldsymbol{0}$};
        \node[state] (D) [thick, below right of=C] {$\boldsymbol{3}$};

        \path (B) edge [thick] node {} (C)
              (B) edge [thick] node {} (F)
              (B) edge [thick] node {} (G)
              (C) edge [thick, bend right = 75, distance=5cm] node {} (F)
              (C) edge [thick] node {} (E)
              (C) edge [thick] node {} (D)
              (C) edge [thick] node {} (G)
              (F) edge [thick] node {} (G)
              (F) edge [thick] node {} (E)
              (E) edge [thick] node {} (G);
    \end{tikzpicture}
    \caption*{\textsc{Stacking-Constraint-Graph}}
\end{figure}


\begin{figure}[H]
  \centering
  \begin{tikzpicture}[scale=0.7, transform shape, node distance=3cm]
        \node[state] (G) [thick] {$\boldsymbol{\textcolor{mygreen}{6}}$};
        \node[state] (B) [thick, above left of=G] {$\boldsymbol{\textcolor{mygreen}{1}}$};
        \node[state] (C) [thick, above right of=G] {$\boldsymbol{\textcolor{mygreen}{2}}$};
        \node[state] (E) [thick, below right of=G] {$\boldsymbol{\textcolor{mygreen}{4}}$};
        \node[state] (F) [thick, below left of=G] {$\boldsymbol{\textcolor{mygreen}{5}}$};
        \node[state] (A) [thick, below left of=B] {$\boldsymbol{\textcolor{red}{0}}$};
        \node[state] (D) [thick, below right of=C] {$\boldsymbol{\textcolor{mygreen}{3}}$};

        \path (B) edge [thick] node {} (C)
              (B) edge [thick] node {} (F)
              (B) edge [thick, mygreen] node {} (G)
              (C) edge [thick, bend right = 75, distance=5cm] node {} (F)
              (C) edge [thick] node {} (E)
              (C) edge [thick, mygreen] node {} (D)
              (C) edge [thick] node {} (G)
              (F) edge [thick] node {} (G)
              (F) edge [thick, mygreen] node {} (E)
              (E) edge [thick] node {} (G);
    \end{tikzpicture}
  \caption*{\textsc{MCM (Item $0$ unmatched)}}
\end{figure}


\begin{figure}[H]
\centering
\begin{tikzpicture}[scale=0.7, transform shape, node distance=3cm]
        \node[state] (A) [thick] {\textcolor{mygreen}{$\boldsymbol{1, 6}$}};
        \node[state] (B) [thick, right of=A] {\textcolor{mygreen}{$\boldsymbol{2, 3}$}};
        \node[state] (C) [thick, right of=B] {\textcolor{mygreen}{$\boldsymbol{4, 5}$}};
        \node[state] (D) [thick, right of=C] {\textcolor{red}{$\boldsymbol{0}$}};
        \path ;
  \end{tikzpicture}
  \caption*{\textsc{Kanten bilden Item-Paare}}
\end{figure}

\begin{figure}[H]
\begin{subfigure}[b]{0.45\textwidth}
\centering
\begin{tikzpicture}[scale=0.7, transform shape, node distance=1.5cm]
        \node[state] (A) [thick] {$\boldsymbol{1, 6}$};
        \node[state] (B) [thick, below of=A] {$\boldsymbol{2, 3}$};
        \node[state] (C) [thick, below of=B] {$\boldsymbol{4, 5}$};
        \node[state] (D) [thick, below of=C] {$\boldsymbol{0}$};

        \node[state] (F) [thick, right = 4cm of A] {$\boldsymbol{S_1}$};
        \node[state] (G) [thick, right = 4cm of B] {$\boldsymbol{S_2}$};
        \node[state] (H) [thick, right = 4cm of C] {$\boldsymbol{S_3}$};
        \node[state] (I) [thick, right = 4cm of D] {$\boldsymbol{S_4}$};

        \path
          % from (1,6) to all stacks
          (A) edge [thick] node {} (F)
          (A) edge [thick] node {} (G)
          (A) edge [thick] node {} (H)
          (A) edge [thick] node {} (I)

          % from (2,3) to all stacks
          (B) edge [thick] node {} (F)
          (B) edge [thick] node {} (G)
          (B) edge [thick] node {} (H)
          (B) edge [thick] node {} (I)

          % from (4,5) to all stacks
          (C) edge [thick] node {} (F)
          (C) edge [thick] node {} (G)
          (C) edge [thick] node {} (H)
          (C) edge [thick] node {} (I)

          % from 0 to all stacks
          (D) edge [thick] node {} (F)
          (D) edge [thick] node {} (G)
          (D) edge [thick] node {} (H)
          (D) edge [thick] node {} (I);
  \end{tikzpicture}
  \caption*{\textsc{vollständig bipartit}}
\end{subfigure}
\hfill
\begin{subfigure}[b]{0.45\textwidth}
\centering
\begin{tikzpicture}[scale=0.7, transform shape, node distance=1.5cm]
        \node[state] (A) [thick] {$\boldsymbol{1, 6}$};
        \node[state] (B) [thick, below of=A] {$\boldsymbol{2, 3}$};
        \node[state] (C) [thick, below of=B] {$\boldsymbol{4, 5}$};
        \node[state] (D) [thick, below of=C] {$\boldsymbol{0}$};

        \node[state] (F) [thick, right = 4cm of A] {$\boldsymbol{S_1}$};
        \node[state] (G) [thick, right = 4cm of B] {$\boldsymbol{S_2}$};
        \node[state] (H) [thick, right = 4cm of C] {$\boldsymbol{S_3}$};
        \node[state] (I) [thick, right = 4cm of D] {$\boldsymbol{S_4}$};

        \path
          (A) edge [thick] node {} (I)
          (B) edge [thick] node {} (H)
          (C) edge [thick] node {} (G)
          (D) edge [thick] node {} (F);
  \end{tikzpicture}
  \caption*{\textsc{MWPM}}
\end{subfigure}
\end{figure}

\begin{figure}[H]
  \centering
  \resizebox{0.4\textwidth}{!}{
    \begin{tabular}{c|c|c|c|c|}
    \cline{2-5}
    $\boldsymbol{L_2}$ & $$ & $5$ & $3$ & $6$ \\ \cline{2-5}
    $\boldsymbol{L_1}$ & $0$ & $4$ & $2$ & $1$ \\ \cline{2-5}
    \multicolumn{1}{c}{} & \multicolumn{1}{c}{$\boldsymbol{S_1}$} & \multicolumn{1}{c}{$\boldsymbol{S_2}$}
    & \multicolumn{1}{c}{$\boldsymbol{S_3}$} & \multicolumn{1}{c}{$\boldsymbol{S_4}$} \\
    \end{tabular}}
    \caption*{\textsc{Zulässige Zuweisungen (ggf. Item-Reihenfolge korrigieren)}}
\end{figure}

\pagebreak

\subsection{Vergleich der $b = 2$ Solver (s)}
\label{sec:solver_comp_b=2_s}

\begin{figure}[H]
\centering
\includegraphics[width=0.3\textwidth]{img/solver_instance_coverage_b=2_s_1s.png}
\includegraphics[width=0.3\textwidth]{img/solver_instance_coverage_b=2_s_3s.png}
\includegraphics[width=0.3\textwidth]{img/solver_instance_coverage_b=2_s_5s.png}
\caption*{\textsc{Zeitlimit} $1s$ $\quad\quad\quad\thinspace\thinspace\thinspace\thinspace\thinspace$ \textsc{Zeitlimit} $3s$
$\thinspace\thinspace\thinspace\thinspace\thinspace\quad\quad\quad$ \textsc{Zeitlimit} $5s$}
% \end{figure}
% \begin{figure}[H]
\begin{subfigure}[b]{0.3\textwidth}
\centering
\resizebox{\textwidth}{!}{
\begin{tabular}{ | l | l | l |}
    \hline
     & \textbf{BinP} & \textbf{3Idx} \\ \hline
    \textbf{Optimal} & $ \textcolor{red}{0 \%}$ & $ \textcolor{mygreen}{35 \%}$ \\ \hline
    \textbf{Laufzeit} & $\textcolor{red}{---}$ & \O $\thinspace \textcolor{mygreen}{0.9 s}$ \\ \hline
    \textbf{Abweichung} & $\textcolor{red}{---}$ & \O $\thinspace \textcolor{mygreen}{2.0 \%}$ \\ \hline
\end{tabular}}
\end{subfigure}
% $\quad\quad\quad\quad$
\begin{subfigure}[b]{0.3\textwidth}
\centering
\resizebox{\textwidth}{!}{
\begin{tabular}{ | l | l | l |}
    \hline
     & \textbf{BinP} & \textbf{3Idx} \\ \hline
    \textbf{Optimal} & $ \textcolor{red}{10 \%}$ & $ \textcolor{mygreen}{100 \%}$ \\ \hline
    \textbf{Laufzeit} & \O $\thinspace \textcolor{red}{3.0 s}$ & \O $\thinspace \textcolor{mygreen}{1.1 s}$ \\ \hline
    \textbf{Abweichung} & \O $\thinspace \textcolor{red}{4.7 \%}$ & \O $\thinspace \textcolor{mygreen}{0.0 \%}$ \\ \hline
\end{tabular}}
\end{subfigure}
% \end{figure}
% \begin{figure}[H]
\begin{subfigure}[b]{0.3\textwidth}
\centering
\resizebox{\textwidth}{!}{
\begin{tabular}{ | l | l | l |}
    \hline
     & \textbf{BinP} & \textbf{3Idx} \\ \hline
    \textbf{Optimal} & $ \textcolor{red}{90 \%}$ & $ \textcolor{mygreen}{100 \%}$ \\ \hline
    \textbf{Laufzeit} & \O $\thinspace \textcolor{red}{3.9 s}$ & \O $\thinspace \textcolor{mygreen}{1.1 s}$ \\ \hline
    \textbf{Abweichung} & \O $\thinspace \textcolor{red}{0.4 \%}$ & \O $\thinspace \textcolor{mygreen}{0.0 \%}$ \\ \hline
\end{tabular}}
\end{subfigure}
\end{figure}
\centering
\textbf{2Cap-Heuristik}\linebreak
\centering
Abweichung vom Optimum: \O $\thinspace \boldsymbol{2.0 \%}$\linebreak
\centering
Laufzeit: \O $\thinspace \boldsymbol{0.02 s}$

\subsection{Vergleich der $b = 2$ Solver (m)}
\label{sec:solver_comp_b=2_m}

\begin{figure}[H]
\centering
\includegraphics[width=0.3\textwidth]{img/solver_instance_coverage_b=2_m_60s.png}
\includegraphics[width=0.3\textwidth]{img/solver_instance_coverage_b=2_m_600s.png}
\includegraphics[width=0.3\textwidth]{img/solver_instance_coverage_b=2_m_1200s.png}
\caption*{\textsc{Zeitlimit} $1min$ $\quad\quad\quad$ \textsc{Zeitlimit} $10min$ $\quad\quad\quad$ \textsc{Zeitlimit} $20min$}
% \end{figure}
% \begin{figure}[H]
% $\quad\quad\quad\quad$
\begin{subfigure}[b]{0.3\textwidth}
\centering
\resizebox{\textwidth}{!}{
\begin{tabular}{ | l | l | l |}
    \hline
     & \textbf{BinP} & \textbf{3Idx} \\ \hline
    \textbf{Optimal} & $ \textcolor{red}{0 \%}$ & $ \textcolor{mygreen}{50 \%}$ \\ \hline
    \textbf{Laufzeit} & $\textcolor{red}{----}$ & \O $\thinspace \textcolor{mygreen}{55.4 s}$ \\ \hline
    \textbf{Abweichung} & $\textcolor{red}{----}$ & \O $\thinspace \textcolor{mygreen}{0.8 \%}$ \\ \hline
\end{tabular}}
\end{subfigure}
\begin{subfigure}[b]{0.3\textwidth}
\centering
\resizebox{\textwidth}{!}{
\begin{tabular}{ | l | l | l |}
    \hline
     & \textbf{BinP} & \textbf{3Idx} \\ \hline
    \textbf{Optimal} & $ \textcolor{red}{10 \%}$ & $ \textcolor{mygreen}{100 \%}$ \\ \hline
    \textbf{Laufzeit} & \O $\thinspace \textcolor{red}{581.6 s}$ & \O $\thinspace \textcolor{mygreen}{73.7 s}$ \\ \hline
    \textbf{Abweichung} & \O $\thinspace \textcolor{red}{0.6 \%}$ & \O $\thinspace \textcolor{mygreen}{0.0 \%}$ \\ \hline
\end{tabular}}
\end{subfigure}
% \end{figure}
% \begin{figure}[H]
\begin{subfigure}[b]{0.3\textwidth}
\centering
\resizebox{\textwidth}{!}{
\begin{tabular}{ | l | l | l |}
    \hline
     & \textbf{BinP} & \textbf{3Idx} \\ \hline
    \textbf{Optimal} & $ 100 \%$ & $ 100 \%$ \\ \hline
    \textbf{Laufzeit} & \O $\thinspace 869.6 s$ & \O $\thinspace \textcolor{mygreen}{107.8 s}$ \\ \hline
    \textbf{Abweichung} & \O $\thinspace 0.0 \%$ & \O $\thinspace 0.0 \%$ \\ \hline
\end{tabular}}
\end{subfigure}
\end{figure}
\centering
\textbf{2Cap-Heuristik}\linebreak
\centering
Abweichung vom Optimum: \O $\thinspace \boldsymbol{0.8 \%}$\linebreak
\centering
Laufzeit: \O $\thinspace \boldsymbol{0.1 s}$

\pagebreak

\subsection{Vergleich der $b = 2$ Solver (l)}
\label{sec:solver_comp_b=2_l}

\begin{figure}[H]
\centering
\includegraphics[width=0.3\textwidth]{img/solver_instance_coverage_b=2_l_900s.png}
\includegraphics[width=0.3\textwidth]{img/solver_instance_coverage_b=2_l_1800s.png}
\includegraphics[width=0.3\textwidth]{img/solver_instance_coverage_b=2_l_2700s.png}
\caption*{\textsc{Zeitlimit} $15min$ $\quad\quad\quad$ \textsc{Zeitlimit} $30min$ $\quad\quad\quad$ \textsc{Zeitlimit} $45min$}
% \end{figure}
% \begin{figure}[H]
\begin{subfigure}[b]{0.3\textwidth}
\centering
\resizebox{\textwidth}{!}{
\begin{tabular}{ | l | l | l |}
    \hline
     & \textbf{BinP} & \textbf{3Idx} \\ \hline
    \textbf{Optimal} & $ \textcolor{red}{0 \%}$ & $ \textcolor{mygreen}{70 \%}$ \\ \hline
    \textbf{Laufzeit} & \textcolor{red}{$---$} & \O $\thinspace \textcolor{mygreen}{728.0 s}$ \\ \hline
    \textbf{Abweichung} & \textcolor{red}{$---$} & \O $\thinspace \textcolor{mygreen}{0.53 \%}$ \\ \hline
\end{tabular}}
\end{subfigure}
\begin{subfigure}[b]{0.3\textwidth}
\centering
\resizebox{\textwidth}{!}{
\begin{tabular}{ | l | l | l |}
    \hline
     & \textbf{BinP} & \textbf{3Idx} \\ \hline
    \textbf{Optimal} & $ \textcolor{red}{0 \%}$ & $ \textcolor{mygreen}{85 \%}$ \\ \hline
    \textbf{Laufzeit} & \textcolor{red}{$---$} & \O $\thinspace \textcolor{mygreen}{977.2 s}$ \\ \hline
    \textbf{Abweichung} & \textcolor{red}{$---$} & \O $\thinspace \textcolor{mygreen}{0.0 \%}$ \\ \hline
\end{tabular}}
\end{subfigure}
% $\quad\quad\quad\quad$
\begin{subfigure}[b]{0.3\textwidth}
\centering
\resizebox{\textwidth}{!}{
\begin{tabular}{ | l | l | l |}
    \hline
     & \textbf{BinP} & \textbf{3Idx} \\ \hline
    \textbf{Optimal} & $ \textcolor{red}{0 \%}$ & $ \textcolor{mygreen}{100 \%}$ \\ \hline
    \textbf{Laufzeit} & \textcolor{red}{$---$} & \O $\thinspace \textcolor{mygreen}{1185.9  s}$ \\ \hline
    \textbf{Abweichung} & \textcolor{red}{$---$} & \O $\thinspace \textcolor{mygreen}{0.0 \%}$ \\ \hline
\end{tabular}}
\end{subfigure}
% \end{figure}
% \begin{figure}[H]
\end{figure}
\centering
\textbf{2Cap-Heuristik}\linebreak
\centering
Abweichung vom Optimum: \O $\thinspace \boldsymbol{0.6 \%}$\linebreak
\centering
Laufzeit: \O $\thinspace \boldsymbol{0.7 s}$

\subsection{Konstruktive Heuristik ($b = 3$)}
\label{sec:three_cap_heuristic}

\begin{itemize}
  \item Item-Paare bilden (\textsc{\textbf{MCM}(items)})
  \item Item-Tripel bilden (\textsc{\textbf{MCM}(pairs, unmatchedItems)})
  \item Verbleibende Item-Paare zu Tripeln mergen
  \item Item-Paare bilden (\textsc{\textbf{MCM}(remainingItems)})
  \item Verbleibende Items als \textquote{unmatched} betrachten
  \item Bipartiten Graph generieren:
  \begin{enumerate}
    \item Items (Item-Tripel, Item-Paare, Unmatched-Items)
    \item Stacks
  \end{enumerate}
  \item \textsc{\textbf{MWPM}} berechnen, Kanten als Stackzuweisungen interpretieren
  \item Ggf. Reihenfolge der Items innerhalb der Stacks anpassen
\end{itemize}

\pagebreak

\textbf{Beispiel}
$\boldsymbol{I} := \{0, 1, 2, 3, 4, 5, 6, 7, 8, 9\} \quad\quad\quad \boldsymbol{b} := 3 \quad\quad\quad \boldsymbol{m} := 4$
\begin{figure}[H]
\centering
\begin{tikzpicture}[scale=0.65, transform shape, node distance=2cm]
        \node[state] (A) [thick] {$\boldsymbol{\textcolor{mygreen}{0, 2}}$};
        \node[state] (B) [thick, right of=A] {$\boldsymbol{\textcolor{mygreen}{1, 9}}$};
        \node[state] (C) [thick, right of=B] {$\boldsymbol{\textcolor{mygreen}{3, 4}}$};
        \node[state] (D) [thick, right of=C] {$\boldsymbol{\textcolor{mygreen}{5, 6}}$};
        \node[state] (E) [thick, right of=D] {$\boldsymbol{\textcolor{red}{7}}$};
        \node[state] (F) [thick, right of=E] {$\boldsymbol{\textcolor{red}{8}}$};
        \path;
  \end{tikzpicture}
  \caption*{\textsc{Item-Paare und Unmatched-Items}}
% \end{figure}
% \begin{figure}[H]
\centering
\begin{tikzpicture}[scale=0.65, transform shape, node distance=2cm]
        \node[state] (A) [thick] {$\boldsymbol{\textcolor{red}{0, 2}}$};
        \node[state] (B) [thick, right of=A] {$\boldsymbol{\textcolor{red}{1, 9}}$};
        \node[state] (C) [thick, right of=B] {$\boldsymbol{\textcolor{mygreen}{3, 4}}$};
        \node[state] (D) [thick, right of =C] {$\boldsymbol{\textcolor{red}{5, 6}}$};
        \node[state] (E) [thick, right of=D] {$\boldsymbol{\textcolor{mygreen}{7}}$};
        \node[state] (F) [thick, right of=E] {$\boldsymbol{\textcolor{red}{8}}$};
        \path
          (E) edge [thick, mygreen, bend right] node {} (C)
          (E) edge [thick] node {} (D);
  \end{tikzpicture}
  \caption*{\textsc{MCM aus Item-Paaren und Unmatched-Items}}
% \end{figure}
% \begin{figure}[H]
\centering
\begin{tikzpicture}[scale=0.65, transform shape, node distance=2cm]
        \node[state] (A) [thick] {$\boldsymbol{3, 4, 7}$};
        \node[state] (B) [thick, right of=A] {$\boldsymbol{0, 2}$};
        \node[state] (C) [thick, right of=B] {$\boldsymbol{1, 9}$};
        \node[state] (D) [thick, right of=C] {$\boldsymbol{5, 6}$};
        \node[state] (E) [thick, right of=D] {$\boldsymbol{8}$};
        \path;
\end{tikzpicture}
\caption*{\textsc{Item-Tripel, Item-Paare und Unmatched-Items}}
\begin{subfigure}[b]{0.4\textwidth}
\centering
\begin{tikzpicture}[scale=0.65, transform shape, node distance=2cm]
        \node[state] (A) [thick] {$\boldsymbol{0, 2}$};
        \node[state] (B) [thick, right of=A] {$\boldsymbol{1, 9}$};
        \node[state] (C) [thick, right of=B] {$\boldsymbol{5, 6}$};
        \path;
  \end{tikzpicture}
  \caption*{\textsc{Item-Paare}}
\end{subfigure}
\hspace{10pt}
\begin{subfigure}[b]{0.4\textwidth}
\centering
\begin{tikzpicture}[scale=0.65, transform shape, node distance=2cm]
        \node[state] (A) [thick] {$\boldsymbol{0, 5, 6}$};
        \node[state] (B) [thick, right of=A] {$\boldsymbol{2, 1, 9}$};
        \path;
  \end{tikzpicture}
  \caption*{\textsc{Merge-Ergebnis}}
\end{subfigure}

\end{figure}

\begin{figure}[H]
% \begin{figure}[H]
\begin{subfigure}[b]{0.4\textwidth}
\centering
\begin{tikzpicture}[scale=0.65, transform shape, node distance=2cm]
        \node[state] (A) [thick] {$\boldsymbol{3, 4, 7}$};
        \node[state] (B) [thick, below of=A] {$\boldsymbol{0, 5, 6}$};
        \node[state] (C) [thick, below of=B] {$\boldsymbol{2, 1, 9}$};
        \node[state] (D) [thick, below of=C] {$\boldsymbol{8}$};

        \node[state] (E) [thick, right = 4cm of A] {$\boldsymbol{S_1}$};
        \node[state] (F) [thick, below of=E] {$\boldsymbol{S_2}$};
        \node[state] (G) [thick, below of=F] {$\boldsymbol{S_3}$};
        \node[state] (H) [thick, below of=G] {$\boldsymbol{S_4}$};

        \path
          (A) edge [thick] node {} (E)
          (A) edge [thick] node {} (F)
          (A) edge [thick] node {} (G)
          (A) edge [thick] node {} (H)

          (B) edge [thick] node {} (E)
          (B) edge [thick] node {} (F)
          (B) edge [thick] node {} (G)
          (B) edge [thick] node {} (H)

          (C) edge [thick] node {} (E)
          (C) edge [thick] node {} (F)
          (C) edge [thick] node {} (G)
          (C) edge [thick] node {} (H)

          (D) edge [thick] node {} (E)
          (D) edge [thick] node {} (F)
          (D) edge [thick] node {} (G)
          (D) edge [thick] node {} (H);
  \end{tikzpicture}
  \caption*{\textsc{Vollständig Bipartit}}
\end{subfigure}
\hfill
\begin{subfigure}[b]{0.4\textwidth}
\centering
\begin{tikzpicture}[scale=0.65, transform shape, node distance=2cm]
        \node[state] (A) [thick] {$\boldsymbol{3, 4, 7}$};
        \node[state] (B) [thick, below of=A] {$\boldsymbol{0, 5, 6}$};
        \node[state] (C) [thick, below of=B] {$\boldsymbol{2, 1, 9}$};
        \node[state] (D) [thick, below of=C] {$\boldsymbol{8}$};

        \node[state] (E) [thick, right = 4cm of A] {$\boldsymbol{S_1}$};
        \node[state] (F) [thick, below of=E] {$\boldsymbol{S_2}$};
        \node[state] (G) [thick, below of=F] {$\boldsymbol{S_3}$};
        \node[state] (H) [thick, below of=G] {$\boldsymbol{S_4}$};

        \path
          (A) edge [thick] node {} (H)
          (B) edge [thick] node {} (F)
          (C) edge [thick] node {} (G)
          (D) edge [thick] node {} (E);
  \end{tikzpicture}
  \caption*{\textsc{MWPM}}
\end{subfigure}
% \end{figure}
\newline\newline
% \begin{figure}[H]
  \centering
  \resizebox{0.35\textwidth}{!}{
    \begin{tabular}{c|c|c|c|c|}
    \cline{2-5}
    $\boldsymbol{L_3}$ & $$ & $6$ & $9$ & $7$ \\ \cline{2-5}
    $\boldsymbol{L_2}$ & $$ & $5$ & $1$ & $4$ \\ \cline{2-5}
    $\boldsymbol{L_1}$ & $8$ & $0$ & $2$ & $3$ \\ \cline{2-5}
    \multicolumn{1}{c}{} & \multicolumn{1}{c}{$\boldsymbol{S_1}$} & \multicolumn{1}{c}{$\boldsymbol{S_2}$}
    & \multicolumn{1}{c}{$\boldsymbol{S_3}$} & \multicolumn{1}{c}{$\boldsymbol{S_4}$} \\
    \end{tabular}}
    \caption*{\textsc{Zulässige Zuweisungen (ggf. Item-Reihenfolge korrigieren)}}
\end{figure}

\pagebreak

\subsection{Vergleich der $b = 3$ Solver (s)}
\label{sec:solver_comp_b=3_s}

\begin{figure}[H]
\centering
\includegraphics[width=0.3\textwidth]{img/solver_instance_coverage_b=3_s_3s.png}
\includegraphics[width=0.3\textwidth]{img/solver_instance_coverage_b=3_s_5s.png}
\includegraphics[width=0.3\textwidth]{img/solver_instance_coverage_b=3_s_10s.png}
\caption*{\textsc{Zeitlimit} $3s$ $\quad\quad\quad\quad$ \textsc{Zeitlimit} $5s$ $\quad\quad\quad\quad$ \textsc{Zeitlimit} $10s$}
% \end{figure}
% \begin{figure}[H]
% $\quad\quad\quad\quad$
\begin{subfigure}[b]{0.3\textwidth}
\centering
\resizebox{\textwidth}{!}{
\begin{tabular}{ | l | l | l |}
    \hline
     & \textbf{BinP} & \textbf{3Idx} \\ \hline
    \textbf{Optimal} & $\textcolor{mygreen}{40 \%}$ & $\textcolor{red}{5 \%}$ \\ \hline
    \textbf{Laufzeit} & \O $\thinspace 2.9 s$ & \O $\thinspace 2.9 s$ \\ \hline
    \textbf{Abweichung} & \O $\thinspace \textcolor{mygreen}{1.7 \%}$ & \O $\thinspace \textcolor{red}{7.8 \%}$ \\ \hline
\end{tabular}}
\end{subfigure}
% \end{figure}
% \begin{figure}[H]
\begin{subfigure}[b]{0.3\textwidth}
\centering
\resizebox{\textwidth}{!}{
\begin{tabular}{ | l | l | l |}
    \hline
     & \textbf{BinP} & \textbf{3Idx} \\ \hline
    \textbf{Optimal} & $\textcolor{mygreen}{100 \%}$ & $\textcolor{red}{50 \%}$ \\ \hline
    \textbf{Laufzeit} & \O $\thinspace \textcolor{mygreen}{3.1 s}$ & \O $\thinspace \textcolor{red}{4.0 s}$ \\ \hline
    \textbf{Abweichung} & \O $\thinspace \textcolor{mygreen}{0.0 \%}$ & \O $\thinspace \textcolor{red}{1.2 \%}$ \\ \hline
\end{tabular}}
\end{subfigure}
\begin{subfigure}[b]{0.3\textwidth}
\centering
\resizebox{\textwidth}{!}{
\begin{tabular}{ | l | l | l |}
    \hline
     & \textbf{BinP} & \textbf{3Idx} \\ \hline
    \textbf{Optimal} & $ \textcolor{mygreen}{100 \%}$ & $ \textcolor{red}{85 \%}$ \\ \hline
    \textbf{Laufzeit} & \O $\thinspace \textcolor{mygreen}{3.1 s}$ & \O $\thinspace \textcolor{red}{5.6 s}$ \\ \hline
    \textbf{Abweichung} & \O $\thinspace \textcolor{mygreen}{0.0 \%}$ & \O $\thinspace \textcolor{red}{0.5 \%}$ \\ \hline
\end{tabular}}
\end{subfigure}
\end{figure}
\centering
\textbf{3Cap-Heuristik}\linebreak
\centering
Abweichung vom Optimum: \O $\thinspace \boldsymbol{2.65 \%}$\linebreak
\centering
Laufzeit: \O $\thinspace \boldsymbol{0.01s}$

\subsection{Vergleich der $b = 3$ Solver (m)}
\label{sec:solver_comp_b=3_m}

\begin{figure}[H]
\centering
\includegraphics[width=0.3\textwidth]{img/solver_instance_coverage_b=3_m_600s.png}
\includegraphics[width=0.3\textwidth]{img/solver_instance_coverage_b=3_m_1200s.png}
\includegraphics[width=0.3\textwidth]{img/solver_instance_coverage_b=3_m_1800s.png}
\caption*{\textsc{Zeitlimit} $10min$ $\quad\quad\quad$ \textsc{Zeitlimit} $20min$ $\quad\quad\quad$ \textsc{Zeitlimit} $30min$}
% \end{figure}
% \begin{figure}[H]
\begin{subfigure}[b]{0.3\textwidth}
\centering
\resizebox{\textwidth}{!}{
\begin{tabular}{ | l | l | l |}
    \hline
     & \textbf{BinP} & \textbf{3Idx} \\ \hline
    \textbf{Optimal} & $ \textcolor{mygreen}{20 \%}$ & $ \textcolor{red}{10 \%}$ \\ \hline
    \textbf{Laufzeit} & \O $\thinspace \textcolor{red}{583.7 s}$ & \O $\thinspace \textcolor{mygreen}{554.3 s}$ \\ \hline
    \textbf{Abweichung} & \O $\thinspace \textcolor{red}{0.2 \%}$ & \O $\thinspace \textcolor{mygreen}{0.01 \%}$ \\ \hline
\end{tabular}}
\end{subfigure}
% $\quad\quad\quad\quad$
\begin{subfigure}[b]{0.3\textwidth}
\centering
\resizebox{\textwidth}{!}{
\begin{tabular}{ | l | l | l |}
    \hline
     & \textbf{BinP} & \textbf{3Idx} \\ \hline
    \textbf{Optimal} & $ \textcolor{mygreen}{100 \%}$ & $ \textcolor{red}{30 \%}$ \\ \hline
    \textbf{Laufzeit} & \O $\thinspace \textcolor{mygreen}{791.4 s}$ & \O $\thinspace \textcolor{red}{795.4 s}$ \\ \hline
    \textbf{Abweichung} & \O $\thinspace 0.0 \%$ & \O $\thinspace 0.0 \%$ \\ \hline
\end{tabular}}
\end{subfigure}
% \end{figure}
% \begin{figure}[H]
\begin{subfigure}[b]{0.3\textwidth}
\centering
\resizebox{\textwidth}{!}{
\begin{tabular}{ | l | l | l |}
    \hline
     & \textbf{BinP} & \textbf{3Idx} \\ \hline
    \textbf{Optimal} & $ \textcolor{mygreen}{100 \%}$ & $ \textcolor{red}{35 \%}$ \\ \hline
    \textbf{Laufzeit} & \O $\thinspace \textcolor{mygreen}{766.4 s}$ & \O $\thinspace \textcolor{red}{874.0 s}$ \\ \hline
    \textbf{Abweichung} & \O $\thinspace 0.0 \%$ & \O $\thinspace 0.0 \%$ \\ \hline
\end{tabular}}
\end{subfigure}
\end{figure}
\centering
\textbf{3Cap-Heuristik}\linebreak
\centering
Abweichung vom Optimum: \O $\thinspace \boldsymbol{1.02 \%}$\linebreak
\centering
Laufzeit: \O $\thinspace \boldsymbol{0.2 s}$

\pagebreak

\subsection{Vergleich der $b = 3$ Solver (l)}
\label{sec:solver_comp_b=3_l}

\begin{figure}[H]
\centering
\includegraphics[width=0.3\textwidth]{img/solver_instance_coverage_b=3_l_1800s.png}
\includegraphics[width=0.3\textwidth]{img/solver_instance_coverage_b=3_l_2700s.png}
\includegraphics[width=0.3\textwidth]{img/na.png}
\caption*{\textsc{Zeitlimit} $30min$ $\quad\quad\quad$ \textsc{Zeitlimit} $45min$ $\quad\quad\quad$ \textsc{Zeitlimit} $60min$}
% \end{figure}
% \begin{figure}[H]
\begin{subfigure}[b]{0.3\textwidth}
\centering
\resizebox{\textwidth}{!}{
\begin{tabular}{ | l | l | l |}
    \hline
     & \textbf{BinP} & \textbf{3Idx} \\ \hline
    \textbf{Optimal} & $ \textcolor{red}{0 \%}$ & $ \textcolor{red}{0 \%}$ \\ \hline
    \textbf{Laufzeit} & $\textcolor{red}{---}$ & $\textcolor{red}{---}$ \\ \hline
    \textbf{Abweichung} & $\textcolor{red}{---}$ & $\textcolor{red}{---}$ \\ \hline
\end{tabular}}
\end{subfigure}
% $\quad\quad\quad\quad$
\begin{subfigure}[b]{0.3\textwidth}
\centering
\resizebox{\textwidth}{!}{
\begin{tabular}{ | l | l | l |}
    \hline
     & \textbf{BinP} & \textbf{3Idx} \\ \hline
    \textbf{Optimal} & $ \textcolor{red}{0 \%}$ & $ \textcolor{red}{0 \%}$ \\ \hline
    \textbf{Laufzeit} & $\textcolor{red}{---}$ & $\textcolor{red}{---}$ \\ \hline
    \textbf{Abweichung} & $\textcolor{red}{---}$ &$\textcolor{red}{---}$ \\ \hline
\end{tabular}}
\end{subfigure}
% \end{figure}
% \begin{figure}[H]
\begin{subfigure}[b]{0.3\textwidth}
\centering
\resizebox{\textwidth}{!}{
\begin{tabular}{ | l | l | l |}
    \hline
     & \textbf{BinP} & \textbf{3Idx} \\ \hline
    \textbf{Optimal} & $-tbd-$ & $-tbd-$ \\ \hline
    \textbf{Laufzeit} & $-tbd-$ & $-tbd-$ \\ \hline
    \textbf{Abweichung} & $-tbd-$ & $-tbd-$ \\ \hline
\end{tabular}}
\end{subfigure}
\end{figure}
\centering
\textbf{3Cap-Heuristik}\linebreak
\centering
Abweichung vom Optimum: $\boldsymbol{tbd}$\linebreak
\centering
Laufzeit: \O $\thinspace \boldsymbol{1.0 s}$


%\section*{Danksagung}
%Optional

%---------------------------------------------------------------------%
% Bibliography
\cleardoublepage
\printbibliography

% Erklaerung, dass man die Arbeit serioes erstellt hat.
\cleardoublepage
\makeatletter
  \hspace{1cm}\vfill{}
	\thispagestyle{empty}

	Ich versichere, dass ich die eingereichte \ThesisType{}
  selbstst\"andig und ohne unerlaubte Hilfe verfasst habe.
  Anderer als der von mir angegebenen Hilfsmittel und Schriften habe ich mich nicht bedient.
  Alle w\"ortlich oder sinngem\"a\ss\ den Schriften anderer Autoren entnommene
  Stellen habe ich kenntlich gemacht.

  \vspace{1cm}

	\begin{flushright}
		Osnabr\"uck, \@date
	\end{flushright}

	\protect{\vspace{0.5cm}}
	\underline{\hspace{7cm}}\\
	(\@author)
\makeatother

\end{document}
